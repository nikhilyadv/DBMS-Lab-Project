\documentclass[a4paper,12pt]{article} 
\usepackage{enumitem}
\usepackage{color}
\usepackage{graphicx}
\usepackage{float}
\usepackage{minted}
\usepackage{hyperref}

% End loading packages

\newcommand{\bck}{
    \textbackslash
}
\newcommand{\iph}[2]{
    \includegraphics[width=#1\textwidth,height=#1\textheight,keepaspectratio]{#2}
}
\newcommand{\ph}[1]{
    \includegraphics[width=0.5\textwidth,height=0.5\textheight,keepaspectratio]{#1}
}
\newcommand{\ita}[1]{
    \textit{#1}
}
\newcommand{\bfa}[1]{
    \textbf{#1}
}
\newcommand{\swastik}[1]{%
    \begin{tikzpicture}[#1]
        \draw (-1,1)  -- (-1,0) -- (1,0) -- (1,-1);
        \draw (-1,-1) -- (0,-1) -- (0,1) -- (1,1);
    \end{tikzpicture}%
}

% End setting basic commands

\setminted{breaklines=true, tabsize=2, breaksymbolleft=}

% End setting basic settings.

\begin{document}
\title{Database Design And Implementation For E-Commerce}
\author{\small{Sourabh Aggarwal (111601025), Nikhil Kumar Yadav (111601013)}}
\date{\today}
\maketitle
\pagenumbering{roman}
\tableofcontents
\newpage
\pagenumbering{arabic}
\section{Contribution}
Please understand that we are a two member team and we both are involved constantly in each phase of the project, be it GUI, Procedures, Report etc. So, the division of task within ourselves isn't that straight forward to describe, just understand that as of now, we did everything together.
\begin{enumerate}
  \item \bfa{Procedures and Functions:}We discussed together about what various procedures and functions we could implement then we decided to divide the task equally among ourselves. So Nikhil wrote some procedures and Sourabh wrote the rest.
\end{enumerate}
\section{Introduction}
In this project, our aim was to come up with a reasonably scalable database along with basic GUI for E-Commerce purpose.

We started by listing down various requirements presented in the next section. After that we proceeed on builing an ER Diagram to fullfil the same. After that it was time to implement all this in SQL. In due time, we managed to put various important features provided by almost all E-Commerce site in our project. 

So the following report touches on each of these aspects in brief and sequential manner. 

\subsection{Requirements}
Following is the list of requirements.

\begin{enumerate}
\item Company maintains the details of stock like their id, name, quantity, rating etc.

\item Company maintains the details of users like their id, name, address, phone number, ewallet.

\item Only users which have purchased the product can leave rating and review to product, they can also give rating to the seller. Thus users should also be able to see their past puchases.

\item Users can add balance to their ewallet.

\item Company maintains the details of its suppliers like their id, name, address, phone number and rating. Each supplier has at least some stock for some item. (Suppliers can add new product (stock) and mention its quantity which he/she has.)

\item When users browses for a product, suppliers will be listed based on the quantity user wants.
\end{enumerate}
\section{Entity Relation Diagram}
\begin{figure}[H]
    \centering
    \includegraphics[width=1\textwidth]{ERD2} 
    \caption{Entity Relationship Diagram for e-commerce}
\end{figure}
\section{Database Schema And Normalization}
In this section we describe various aspects of our schema and also mention that it is indeed normalized (BCNF).

\ita{Notation: }A dependency $A \rightarrow B$ is called relevant if all other dependencies from $A$ are of the form $A \rightarrow C$ where $C \subseteq B$.

\begin{itemize}
  \item A table for basic details of customer.
  \begin{minted}{SQL}
/*  Here: customer_id -> R is the only relevant dependency and hence it is in BCNF */
create table customer (
  customer_id VARCHAR (20) primary key not null,
  name VARCHAR (20) not null,
  address VARCHAR (60) not null,
  phone_number DECIMAL (10) UNSIGNED not null,
  email_id VARCHAR (20) not null
);
  \end{minted}
  \item A table for basic details of seller.
  \begin{minted}{SQL}
    /* Here: seller_id -> R is the only relevant dependency and hence it is in BCNF */
    /* Rating will be updated with the help of triggers. */
    create table seller (
      seller_id varchar (20) primary key not null,
      name varchar (20) not null,
      address varchar (60) not null,
      phone_number decimal (10) UNSIGNED NOT NULL,
      email_id VARCHAR (20) not null,
      rating float
    );
  \end{minted}
  \item A table for basic details of shipper.
  \begin{minted}{SQL}
    /*  Here: shipper_id -> R is the only relevant dependency and hence it is in BCNF  */
create table shipper (
  shipper_id varchar (20) primary key not null,
  name varchar (20) not null,
  head_quarters varchar (60) not null,
  phone_number decimal (10) UNSIGNED not null,
  email_id VARCHAR (20) not null
);

  \end{minted}
  \item Having described these basic tables, we can now describe table for products. Note that each product can be sold by different sellers in different price and quantity, thus, primary key is formed by both product\_id and seller\_id. 
  \begin{minted}{SQL}
    /*  Here: (product_id, seller_id) -> R is the only relevant dependency and hence it is in BCNF  */
    /* Rating will be updated with the help of triggers. */
    create table product (
      product_id varchar (20) not null,
      product_name varchar (20) not null,
      seller_id varchar (20) not null,
      price float not NULL,
      total_stock int,
      pickup_address varchar (60) not null,
      description varchar (60),
      rating float,
      foreign key (seller_id) references seller (seller_id) on delete cascade,
      primary key (product_id, seller_id)
    );
  \end{minted}
  \item When user makes a payment, we want to store payment details for which we have the following table.
  \begin{minted}{SQL}
/*   Here: payment_id -> R is the only relevant dependency and hence it is in BCNF  */
create table payment (
  payment_id VARCHAR (20) primary key not null,
  credit_card_number VARCHAR (20) not null,
  date_ timestamp,
  billing_address varchar(60) not null
);
  \end{minted}
  \item User will have a front end feature to add items in cart. When the user is ready to buy, it will generate an \ita{order\_id} for all those products which he or she chose. Note that \ita{order\_id} will be generated \ita{only} when user successfully does the payment.
  \begin{minted}{SQL}
    /* Here: order_id -> R is the only relevant dependency and hence it is in BCNF  */
    create table order_ (
      order_id VARCHAR (20) primary key not null,
      customer_id VARCHAR (20),
      shipping_address varchar(60) not null,
      payment_id VARCHAR (20),
      foreign key (customer_id) references customer (customer_id) on delete set null,
      foreign key (payment_id) references payment (payment_id) on delete set null
    );
  \end{minted}
  \item After generating the payment, we have to put the details of the bought items along with their order\_id.
  \begin{minted}{SQL}
    /*  Here: (product_id, order_id, seller_id) -> R is the only relevant dependency and hence it is in BCNF  */
    create table product_order (
      product_id varchar(20) not null,
      order_id varchar (20) not null,
      seller_id varchar (20),
      product_rating int check (product_rating in (NULL, 1, 2, 3, 4, 5)),
      seller_rating int check (seller_rating in (NULL, 1, 2, 3, 4, 5)),
      ship_index int,
      product_review varchar (60),
      seller_review varchar (60),
      quantity int,
      selling_price float,
      primary key (product_id, order_id, seller_id),
      foreign key (product_id) references product (product_id) on delete cascade,
      foreign key (order_id) references order_ (order_id) on delete cascade,
      foreign key (seller_id) references seller (seller_id) on delete cascade,
      foreign key (ship_index) references track (index_) on delete set null
    );
  \end{minted}
  \item Note that we used a foreign key in the above table which we haven't defined yet, which is \ita{ship\_index}. It is basically a unique identifier for each ordered product serving as an index of track table which we will use to track our items.
  \begin{minted}{SQL}
    /*  Here: index_ -> R is the only relevant dependency and hence it is in BCNF  */
    create table track (
      index_ INT AUTO_INCREMENT primary key not null,
      shipper_id varchar (20),
      tracking_id varchar (20),
      foreign key (shipper_id) references shipper (shipper_id) on delete set null
    );
  \end{minted}
  \item We also have an auxiliary table for keeping track of users with their old passwords as mysql.user encrypts the passwords and there is no way to get it back also this table is required for validation when the user tries to log in the system.
  \begin{minted}{SQL}

-- Clearly this table is in BCNF.
create table Users (
  username VARCHAR (20) primary key not null,
  passcode VARCHAR (20) not null,
  roles VARCHAR (20) not null
);
  \end{minted}
\end{itemize}
\newpage
\section{Roles, Triggers, Views}
\subsection{Views}
\begin{minted}{SQL}
  -- #########################################
-- ###########CUSTOMER VIEWS################
-- #########################################

-- This view will allow customer to view its details.
CREATE VIEW customer_add AS (SELECT * 
                             FROM customer 
                             WHERE CONCAT(customer_id, "@localhost") IN (SELECT user()));

-- This view will allow customer to see the total cost of his/her various orders
CREATE VIEW orderPrice AS (SELECT order_id, sum(selling_price * quantity) as total_price
                           FROM (product_order)
                           GROUP BY order_id); 

-- This view will tell the customer details corresponding to his/her all order_id mentioning complete order details (order_id, shipping_address, date_, total_price) except the products in that order. 
CREATE VIEW previousOrders AS (SELECT T1.order_id, T1.shipping_address, T2.date_, T3.total_price
                               FROM (order_ as T1) NATURAL JOIN (payment as T2) NATURAL JOIN (orderPrice as T3)
                               WHERE CONCAT(T1.customer_id, "@localhost") IN (SELECT user()));

-- This view will allow customer to just see his various order_id. 
CREATE VIEW listOrders AS (SELECT order_id
                            FROM order_ 
                            WHERE CONCAT(customer_id, "@localhost") IN (SELECT user()));

-- This view will give entry to the track table for each (product_id, order_id) pair
CREATE VIEW trackID AS (SELECT order_id, product_id, ship_index
                        FROM product_order
                        WHERE order_id IN (SELECT * FROM listOrders));

-- This view will augment the previous view with tracking_id as well.
CREATE VIEW packageStatus AS (SELECT T1.order_id, T1.product_id, T1.ship_index, T2.tracking_id
                              FROM (trackID as T1) JOIN (track as T2) ON (T1.ship_index = T2.index_));


-- #########################################
-- ###########SELLER VIEWS##################
-- #########################################

-- This view will allow seller to see his/her various products.
CREATE VIEW sellerProducts AS (SELECT product_id, product_name, price, total_stock, pickup_address, description 
                                  FROM product
                                  WHERE CONCAT(seller_id, "@localhost") in (SELECT user()));

-- This view allow seller to see various orders which he or she have sold (seller_id, product_id, quantity, selling_price, date_) 
CREATE VIEW sellerOrders AS (SELECT T1.seller_id, T1.product_id, T1.quantity, T1.selling_price, T2.date_
                                FROM (product_order as T1) natural join (payment as T2)
                                WHERE CONCAT(T1.seller_id, "@localhost") in (SELECT user()));

-- #########################################
-- ###########SHIPPER VIEWS#################
-- #########################################

-- This view will allow shipper details (pickup_address, shipping_address, tracking_id)
CREATE VIEW shipperTrack AS (SELECT index_, pickup_address AS source, shipping_address AS destination, tracking_id
                              FROM (track JOIN product_order ON index_ = ship_index) NATURAL JOIN order_ NATURAL JOIN product 
                              WHERE CONCAT(shipper_id, "@localhost") = (SELECT user()));
\end{minted}

\subsection{Roles}
Basically we have three roles.

\begin{itemize}
    \item A role for database administrator.
    \item A role for customer.
    \item A role for supplier.
    \item A role for shipper.
\end{itemize}

And their details is best understood with the help of the following code:
\begin{minted}{SQL}
  CREATE ROLE dbadmin;
CREATE ROLE customer;
CREATE ROLE seller;
CREATE ROLE shipper;

GRANT ALL PRIVILEGES ON AmaKart.* TO dbadmin;

-- Make sure that any view on which a role gets access on should have the filter "SELECT user()"
GRANT ALL PRIVILEGES ON AmaKart.customer_add TO customer;
GRANT SELECT ON AmaKart.previousOrders TO customer;
GRANT SELECT ON AmaKart.listOrders TO customer;
GRANT SELECT ON AmaKart.packageStatus TO customer;

GRANT SELECT ON AmaKart.sellerProducts TO seller;
GRANT SELECT ON AmaKart.sellerOrders TO seller;

GRANT SELECT ON AmaKart.shipperTrack TO shipper;
\end{minted}

\subsection{Triggers}
\begin{minted}{SQL}
-- When a product is sold, we want to mention its selling_price as later the seller can update the price
DELIMITER //
CREATE TRIGGER setPrice BEFORE INSERT on product_order
FOR EACH ROW BEGIN
  SET NEW.selling_price = (SELECT price FROM product WHERE product_id = NEW.product_id and seller_id = NEW.seller_id);
END//
DELIMITER ;

-- When a customer passes a rating for product we have to update it in our product table
DELIMITER //
CREATE TRIGGER updateRatingProduct AFTER UPDATE on product_order
FOR EACH ROW BEGIN
  IF NEW.product_rating != NULL THEN
    UPDATE product SET rating = (SELECT AVG(product_rating) FROM product_order WHERE product_id = NEW.product_id) WHERE product_id = NEW.product_id;
  END IF;
END//
DELIMITER ;


-- When a customer passes a rating for seller we have to update it in our seller table
DELIMITER //
CREATE TRIGGER updateRatingSeller AFTER UPDATE on product_order
FOR EACH ROW BEGIN
  IF NEW.seller_rating != NULL THEN
    UPDATE seller SET rating = (SELECT AVG(seller_rating) FROM product_order WHERE seller_id = NEW.seller_id) WHERE seller_id = NEW.seller_id;
  END IF;
END//
DELIMITER ;

-- When a product is sold, we need to add an entry to our track table for the same
DELIMITER //
CREATE TRIGGER addTrack BEFORE INSERT on product_order
FOR EACH ROW BEGIN
  INSERT INTO track () Values ();
  SET NEW.ship_index = (SELECT MAX (index_) FROM track);
END//
DELIMITER ;
\end{minted}

\newpage
\section{Functions And Procedures}
Below you can see details description and implementation of various Procedures and Functions
\begin{minted}{SQL}
  -- #########################################
  -- ###########CUSTOMER PROCEDURES###########
  -- #########################################
  
  -- Procedure for customer to see his or her past purchases within a specific time duration
  DELIMITER //
  CREATE PROCEDURE seePurchasesBetweenDuration(IN startTime TIMESTAMP, IN endTime TIMESTAMP)
  BEGIN
      select * from payment natural join order_ natural join product_order where CONCAT(order_.customer_id, "@localhost") IN (SELECT user()) AND payment.date_ BETWEEN startTime AND endTime;
  END;
  //
  DELIMITER ;
  
  -- Procedure to see latest N Purchases
  DELIMITER //
  CREATE PROCEDURE seeLatestNPurchases(IN N INT)
  BEGIN
      select * from payment natural join order_ natural join product_order where CONCAT(order_.customer_id, "@localhost") IN (SELECT user()) ORDER BY payment.date_ DESC LIMIT N;
  END;
  //
  DELIMITER ;
  
  -- Procedure to see products within price range
  DELIMITER //
  CREATE PROCEDURE queryProductsTim(IN productName varchar(20), IN lowRange FLOAT, IN highRange FLOAT)
  BEGIN
      select * from product where product_name like CONCAT('%', productName, '%') AND price BETWEEN lowRange AND highRange ORDER BY price ASC;
  END;
  //
  DELIMITER ;
  
  -- Procedure to see reviews of a product withing a duration
  DELIMITER //
  CREATE PROCEDURE recentProductReviewsBetweenDuration(IN pid varchar(20), IN sid varchar(20), IN startTime TIMESTAMP, IN endTime TIMESTAMP)
  BEGIN
      SELECT name, product_review FROM (product_order natural join order_ natural join customer natural join payment) WHERE product_id = pid AND seller_id = sid AND (payment.date_ BETWEEN startTime AND endTime);
  END;
  //
  DELIMITER ;
  
  -- Procedure to add review for a product
  DELIMITER //
  CREATE PROCEDURE addReviewProduct(IN pid varchar(20), IN oid varchar(20), IN rev varchar(60))
  BEGIN
      UPDATE product_order SET product_review = rev WHERE product_id = pid and order_id = oid;
  END;
  //
  DELIMITER ;
  
  -- Procedure to add review for a seller 
  DELIMITER //
  CREATE PROCEDURE addReviewSeller(IN pid varchar(20), IN oid varchar(20), IN rev varchar(60))
  BEGIN
      UPDATE product_order SET seller_review = rev WHERE product_id = pid and order_id = oid;
  END;
  //
  DELIMITER ;
  
  -- Procedure to see products sorted by rating
  DELIMITER //
  CREATE PROCEDURE queryProductsRat(IN productName varchar(20))
  BEGIN
      select * from product where product_name like CONCAT('%', productName, '%') ORDER BY rating DESC;
  END;
  //
  DELIMITER ;
  
  -- Procedure to see reviews of a product withing a duration
  DELIMITER //
  CREATE PROCEDURE recentProductReviewsBetweenDuration(IN pid varchar(20), IN sid varchar(20), IN startTime TIMESTAMP, IN endTime TIMESTAMP)
  BEGIN
      SELECT name, product_review FROM (product_order natural join order_ natural join customer natural join payment) WHERE product_id = pid AND seller_id = sid AND (payment.date_ BETWEEN startTime AND endTime);
  END;
  //
  DELIMITER ;
  
  -- Procedure to add review for a product
  DELIMITER //
  CREATE PROCEDURE addReviewProduct(IN pid varchar(20), IN oid varchar(20), IN rev varchar(60))
  BEGIN
      UPDATE product_order SET product_review = rev WHERE product_id = pid and order_id = oid;
  END;
  //
  DELIMITER ;
  
  -- Procedure to add review for a seller 
  DELIMITER //
  CREATE PROCEDURE addReviewSeller(IN pid varchar(20), IN oid varchar(20), IN rev varchar(60))
  BEGIN
      UPDATE product_order SET seller_review = rev WHERE product_id = pid and order_id = oid;
  END;
  //
  DELIMITER ;
  
  -- Procedure to add rating for product
  DELIMITER //
  CREATE PROCEDURE addRatingProduct(IN pid varchar(20), IN oid varchar(20), IN rating INT)
  BEGIN
      IF (rating IN (1,2,3,4,5)) THEN
        UPDATE product_order SET product_rating =  rating WHERE product_id = pid and order_id = oid;
      END IF;
  END;
  //
  DELIMITER ;
  
  -- Procedure to add rating for seller
  DELIMITER //
  CREATE PROCEDURE addRatingSeller(IN pid varchar(20), IN oid varchar(20), IN rating INT)
  BEGIN
      IF (rating IN (1,2,3,4,5)) THEN
        UPDATE product_order SET seller_rating = rating WHERE product_id = pid and order_id = oid;
      END IF;
  END;
  //
  DELIMITER ;
  
  -- #########################################
  -- ###########SELLER   PROCEDURES###########
  -- #########################################
  
  -- Procedure for seller to see his or her past sold products within a specific time duration
  DELIMITER //
  CREATE PROCEDURE seeSellingsBetweenDuration(IN startTime TIMESTAMP, IN endTime TIMESTAMP)
  BEGIN
      select * from product where (product_id, seller_id) in (select product_order.product_id, product_order.seller_id from payment natural join order_ natural join product_order where CONCAT(product_order.seller_id, "@localhost") IN (SELECT user()) AND payment.date_ BETWEEN startTime AND endTime);
  END;
  //
  DELIMITER ;
  
  -- Procedure to see latest N Sellings
  DELIMITER //
  CREATE PROCEDURE seeLatestNSellings(IN N INT)
  BEGIN
      select * from product where (product_id, seller_id) in (select product_id, seller_id from payment natural join order_ natural join product_order where CONCAT(product_order.seller_id, "@localhost") IN (SELECT user()) ORDER BY payment.date_ DESC) LIMIT N;
  END;
  //
  DELIMITER ;
  
  -- Procedure to see similary products with increasing price
  DELIMITER //
  CREATE PROCEDURE selQuerySimProducts(IN productName varchar(20))
  BEGIN
      select * from product where product_name like CONCAT('%', productName, '%') AND CONCAT(seller_id, "@localhost") IN (SELECT user()) ORDER BY price ASC;
  END;
  //
  DELIMITER ;
  
  -- Procedure to see similar products sorted by rating
  DELIMITER //
  CREATE PROCEDURE selQueryProductsRat(IN productName varchar(20))
  BEGIN
      select * from product where product_name like CONCAT('%', productName, '%') AND CONCAT(seller_id, "@localhost") IN (SELECT user()) ORDER BY rating DESC;
  END;
  //
  DELIMITER ;
  
  -- #########################################
  -- ###########SHIPPER PROCEDURES############
  -- #########################################
  
  -- Procedure for shipper to see his or her past shipments within a specific time duration
  DELIMITER //
  CREATE PROCEDURE seeShipmentsBetweenDuration(IN startTime DATE, IN endTime DATE)
  BEGIN
      select * from track where CONCAT(shipper_id, "@localhost") IN (SELECT user()) AND date_ BETWEEN startTime AND endTime;
  END;
  //
  DELIMITER ;
  
  -- Procedure to see latest N Shipments
  DELIMITER //
  CREATE PROCEDURE seeLatestNShipments(IN N INT)
  BEGIN
      select * from track where CONCAT(shipper_id, "@localhost") IN (SELECT user()) ORDER BY date_ DESC LIMIT N;
  END;
  //
  DELIMITER ;
  
  -- #########################################
  -- ################FUNCTIONS################
  -- #########################################
  
  -- Function to return the total earning of a seller between supplied dates
  DELIMITER //
  CREATE FUNCTION sellerStatsBetweenDate(startTime TIMESTAMP, endTime TIMESTAMP)
  RETURNS FLOAT DETERMINISTIC  
  BEGIN
      DECLARE temp FLOAT;
      SELECT count(quantity*selling_price) INTO temp FROM product_order natural join payment WHERE date_ BETWEEN startTime and endTime;
      RETURN temp;
  END;
  //
  DELIMITER ;
  
\end{minted}
\section{Use Cases}
Lets for this case assume that we a supplier. We first want to register and then add products to sell.
\begin{figure}[H]
    \centering
    \includegraphics[scale=0.5]{login1.png} 
    \caption{Login page.}
\end{figure}
So at the login page we will select SignUp. It will then open up the SignUp page.
\begin{figure}[H]
    \centering
    \includegraphics[scale=0.5]{signup1.png}
    \caption{SignUp page(Selecting role).}
\end{figure}
Since we are Supplier we will select supplier option and then proceed.
\begin{figure}[H]
    \centering
    \includegraphics[scale=0.5]{signup2.png}
    \caption{SignUp page(Entering Details).}
\end{figure}
After entering the details press the Register button. It will display that the user is successfully created.
\begin{figure}[H]
    \centering
    \includegraphics[scale=0.5]{signup3.png}
    \caption{SignUp page(Conformation).}
\end{figure}
After registering goto the login page by pressing the Switch to Login button.
\begin{figure}[H]
    \centering
    \includegraphics[scale=0.5]{login2.png} 
    \caption{Login page(Entering Details).}
\end{figure}
After login you will be taken to a welcome page which will have two options for you 
\begin{itemize}
    \item Either to add new products in the market.
    \item Or to change the quantites or price of existing products. 
\end{itemize}
\begin{figure}[H]
    \centering
    \includegraphics[scale=0.5]{supp1.png} 
    \caption{Welcome page for Supplier.}
\end{figure}
So lets add a new product. You will be taken to a page which will ask to fill the details of the product.
\begin{figure}[H]
    \centering
    \includegraphics[scale=0.5]{supp2.png} 
    \caption{Add new product.}
\end{figure}
Now the product has been added successfully.

\newpage
\section{Useful links}
\begin{center}
\textbf{Complete project - } \href{https://github.com/nikhilyadv/DBMS-Lab-Project}{Link}\\
\textbf{GUI source code - } \href{https://github.com/nikhilyadv/DBMS-Lab-Project/tree/master/GUI}{Link}
\end{center}
\end{document}